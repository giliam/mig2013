%\definecolor{Rose}{rgb}{0.94,0,0.94}
%\definecolor{Gris}{rgb}{0.63,0.63,0.63}
%\definecolor{Bleu}{rgb}{0,0,0.63}
%\definecolor{BleuDeux}{rgb}{0,0,1}
%\definecolor{Vert}{rgb}{0,0.5,0}
%
%\newcommand{\SetCodeStyle}[1]{
%	\ifthenelse{\equal{#1}{C++}}
%	{
%		\lstset{ %
%		  language=C++,
%		  keywordstyle=\color{Bleu},       % keyword style
%		  commentstyle=\color{Gris},    % comment style
%		  numberstyle=\tiny\color{Rose}, % the style that is used for the line-numbers
%		  stringstyle=\color{BleuDeux},     % string literal style
%		  backgroundcolor=\color{white},   % choose the background color; you must add \usepackage{color} or \usepackage{xcolor}
%		  breakatwhitespace=true,          % sets if automatic breaks should only happen at whitespace
%		  breaklines=true,                 % sets automatic line breaking
%		  frame=single,                    % adds a frame around the code
%		  keepspaces=true,                 % keeps spaces in text, useful for keeping indentation of code (possibly needs columns=flexible)
%		  morekeywords={malloc, cDouble},           % if you want to add more keywords to the set
%		  numbers=left,                    % where to put the line-numbers; possible values are (none, left, right)
%		  numbersep=5pt,                   % how far the line-numbers are from the code
%		  tabsize=4,                       % sets default tabsize to 2 spaces
%		}
%	}
%	{
%		\lstset{ %
%		  language=Python,
%		  %literate={\{}{{\color{Rouge}{\{}}}1
%       %                                   {\}}{{\color{Rouge}{\}}}}1
%        %                                  {(}{{\color{Rouge}{(}}}1
%         %                                 {)}{{\color{Rouge}{)}}}1
%          %                                {>}{{\color{Rouge}{$>$}}}1
%           %                               {=}{{\color{Rouge}{$=$}}}1
%            %                              {;}{{\color{Rouge}{$;$}}}1, % Operateurs C++
%		  keywordstyle=\color{Bleu},       % keyword style
%		  commentstyle=\color{Vert},    % comment style
%		  numberstyle=\tiny\color{Vert}, % the style that is used for the line-numbers
%		  stringstyle=\color{Gris},     % string literal style
%		  backgroundcolor=\color{white},   % choose the background color; you must add \usepackage{color} or \usepackage{xcolor}
%		  breakatwhitespace=true,          % sets if automatic breaks should only happen at whitespace
%		  breaklines=true,                 % sets automatic line breaking
%		  frame=single,                    % adds a frame around the code
%		  keepspaces=true,                 % keeps spaces in text, useful for keeping indentation of code (possibly needs columns=flexible)
%		  morekeywords={as},           % if you want to add more keywords to the set
%		  numbers=left,                    % where to put the line-numbers; possible values are (none, left, right)
%		  numbersep=5pt,                   % how far the line-numbers are from the code
%		  tabsize=4,                       % sets default tabsize to 2 spaces
%		}
%	}
%}

\part{Core}
	\setcounter{chapter}{0}
	\nonewpagechapter{Code Principal}
	\section{shell.py}
	\lstinputlisting[language=Python, firstline=20, lastline=318]{../src/shell.py}
	
	\section{server.py}
	\lstinputlisting[language=Python, firstline=9, lastline=21]{../src/server.py}
	
	\section{gui.py}
	\lstinputlisting[language=Python, firstline=17, lastline=247]{../src/gui.py}
	\chapter{handling}
	\section{fenetrehann.py}
	\lstinputlisting[language=Python, firstline=9, lastline=45]{../src/core/handling/fenetrehann.py}
	
	\section{inverseDCT.py}
	\lstinputlisting[language=Python, firstline=9, lastline=39]{../src/core/handling/inverseDCT.py}
	
	\section{triangularFilterbank.py}
	\lstinputlisting[language=Python, firstline=7, lastline=29]{../src/core/handling/triangularFilterbank.py}
	
	\section{passehaut.py}
	\lstinputlisting[language=Python, firstline=9, lastline=15]{../src/core/handling/passehaut.py}
	
	\section{fft.cpp}
	\lstinputlisting[language=C++, firstline=6, lastline=132]{../src/core/handling/fft.cpp}
	\chapter{HMM}
	\section{creationVecteurHMM.py}
	\lstinputlisting[language=Python, firstline=4, lastline=47]{../src/core/hmm/creationVecteurHMM.py}
	
	\section{markov.py}
	\lstinputlisting[language=Python, firstline=8, lastline=237]{../src/core/hmm/markov.py}
	
	\section{tableauEnergyPerFrame.py}
	\lstinputlisting[language=Python, firstline=4, lastline=8]{../src/core/hmm/tableauEnergyPerFrame.py}
	
	\section{hmm.cpp}
	\lstinputlisting[language=C++, firstline=7, lastline=885]{../src/core/hmm/hmm.cpp}
	\chapter{recorder}
	\section{recorder.py}
	\lstinputlisting[language=Python, firstline=13, lastline=64]{../src/core/recording/recorder.py}
	
	\section{sync.py}
	\lstinputlisting[language=Python, firstline=9, lastline=128]{../src/core/recording/sync.py}
	\chapter{utils}
	\section{animate.py}
	\lstinputlisting[language=Python, firstline=8, lastline=33]{../src/core/utils/animate.py}
	
	\section{constantes.py}
	\lstinputlisting[language=Python, firstline=6, lastline=26]{../src/core/utils/constantes.py}
	
	\section{db.py}
	\lstinputlisting[language=Python, firstline=9, lastline=252]{../src/core/utils/db.py}
	
	\section{util.py}
	\lstinputlisting[language=Python, firstline=14, lastline=40]{../src/core/utils/util.py}
	
	
\part{SpeechApp}
	\setcounter{chapter}{0}
	\lstdefinelanguage{JavaScript}
{
  keywords={typeof, new, true, false, catch, function, return, null, catch, switch, var, if, in, while, do, else, case, break},
  %keywordstyle=\color{blue}\bfseries,
  ndkeywords={class, export, boolean, throw, implements, import, this},
  %ndkeywordstyle=\color{darkgray}\bfseries,
  %identifierstyle=\color{black},
  sensitive=false,
  comment=[l]{//},
  morecomment=[s]{/*}{*/},
  %commentstyle=\color{purple}\ttfamily,
  %stringstyle=\color{red}\ttfamily,
  morestring=[b]',
  morestring=[b]"
}

	\section{main}
	\lstinputlisting[language=JavaScript, firstline=7, lastline=270, breaklines=true, frame=single]{../src/speechapp/js/main.js}
	
	\section{holder}
	\lstinputlisting[language=JavaScript, firstline=11, lastline=500, breaklines=true, frame=single]{../src/speechapp/js/holder.js}
	
	\section{recorder}
	\lstinputlisting[language=JavaScript, firstline=1, lastline=87, breaklines=true, frame=single]{../src/speechapp/js/recorder.js}
	
	\section{recorderWorker}
	\lstinputlisting[language=JavaScript, firstline=1, lastline=131, breaklines=true, frame=single]{../src/speechapp/js/recorderWorker.js}
	\section{index.html}
	\lstinputlisting[language=HTML, firstline=1, lastline=147]{../src/speechapp/index.html}
	
\part{SpeechServer}
	\setcounter{chapter}{0}
	\nonewpagechapter{SpeechServer}
	\section{main.py}
	\lstinputlisting[language=Python, firstline=20, lastline=104, breaklines=true, frame=single]{../src/speechserver/main.py}
	
	\section{audioConverter.py}
	\lstinputlisting[language=Python, firstline=14, lastline=141, breaklines=true, frame=single]{../src/speechserver/audioConverter.py}
	
	\section{clientAuth.py}
	\lstinputlisting[language=Python, firstline=7, lastline=103, breaklines=true, frame=single]{../src/speechserver/clientAuth.py}
	
	\section{speechActions.py}
	\lstinputlisting[language=Python, firstline=7, lastline=50, breaklines=true, frame=single]{../src/speechserver/speechActions.py}
