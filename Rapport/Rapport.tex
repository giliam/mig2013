\documentclass[a4paper,12pt]{report} % Rapport de MIG SE 2013

\usepackage[frenchb]{babel} %Packages linguistiques
\usepackage[T1]{fontenc}

\usepackage{amsmath} %Packages math�matiques
\usepackage{amsfonts}
\usepackage{textcomp}

\usepackage[a4paper,left=2cm,right=2cm,top=2cm,bottom=2cm]{geometry}

\usepackage{listings} %Pour le code
\usepackage{ifthen}
\usepackage{color}

\definecolor{Rose}{rgb}{0.94,0,0.94}
\definecolor{Gris}{rgb}{0.63,0.63,0.63}
\definecolor{Bleu}{rgb}{0,0,0.63}
\definecolor{BleuDeux}{rgb}{0,0,1}
\definecolor{Vert}{rgb}{0,0.5,0}

\newcommand{\SetCodeStyle}[1]{
	\ifthenelse{\equal{#1}{C++}}
	{
		\lstset{ %
		  language=C++,
		  keywordstyle=\color{Bleu},       % keyword style
		  commentstyle=\color{Gris},    % comment style
		  numberstyle=\tiny\color{Rose}, % the style that is used for the line-numbers
		  stringstyle=\color{BleuDeux},     % string literal style
		  backgroundcolor=\color{white},   % choose the background color; you must add \usepackage{color} or \usepackage{xcolor}
		  breakatwhitespace=true,          % sets if automatic breaks should only happen at whitespace
		  breaklines=true,                 % sets automatic line breaking
		  frame=single,                    % adds a frame around the code
		  keepspaces=true,                 % keeps spaces in text, useful for keeping indentation of code (possibly needs columns=flexible)
		  morekeywords={malloc, cDouble},           % if you want to add more keywords to the set
		  numbers=left,                    % where to put the line-numbers; possible values are (none, left, right)
		  numbersep=5pt,                   % how far the line-numbers are from the code
		  tabsize=4,                       % sets default tabsize to 2 spaces
		}
	}
	{
		\lstset{ %
		  language=Python,
		  %literate={\{}{{\color{Rouge}{\{}}}1
       %                                   {\}}{{\color{Rouge}{\}}}}1
        %                                  {(}{{\color{Rouge}{(}}}1
         %                                 {)}{{\color{Rouge}{)}}}1
          %                                {>}{{\color{Rouge}{$>$}}}1
           %                               {=}{{\color{Rouge}{$=$}}}1
            %                              {;}{{\color{Rouge}{$;$}}}1, % Operateurs C++
		  keywordstyle=\color{Bleu},       % keyword style
		  commentstyle=\color{Vert},    % comment style
		  numberstyle=\tiny\color{Vert}, % the style that is used for the line-numbers
		  stringstyle=\color{Gris},     % string literal style
		  backgroundcolor=\color{white},   % choose the background color; you must add \usepackage{color} or \usepackage{xcolor}
		  breakatwhitespace=true,          % sets if automatic breaks should only happen at whitespace
		  breaklines=true,                 % sets automatic line breaking
		  frame=single,                    % adds a frame around the code
		  keepspaces=true,                 % keeps spaces in text, useful for keeping indentation of code (possibly needs columns=flexible)
		  morekeywords={as},           % if you want to add more keywords to the set
		  numbers=left,                    % where to put the line-numbers; possible values are (none, left, right)
		  numbersep=5pt,                   % how far the line-numbers are from the code
		  tabsize=4,                       % sets default tabsize to 2 spaces
		}
	}
}
\title{[MIG] Syst�mes Embarqu�s}
\author{Julien \bsc{Caillard}, Adrien \bsc{De La Vaissi�re}, Thomas \bsc{Debarre},\\ Matthieu \bsc{Denoux}, Maxime \bsc{Ernoult}, Axel \bsc{Goering},\\ Cl�ment \bsc{Joudet}, Nathana�l \bsc{Kasriel}, Anis \bsc{Khlif},\\ Sofiane \bsc{Mahiou}, Paul \bsc{Musti�re}, Cl�ment \bsc{Roig}, David \bsc{Vitoux}}
\date{18/11/13 - 6/12/13}

\begin{document}
\maketitle
\tableofcontents

\part{La reconnaissance vocale}
    \chapter{Le traitement du signal}
        \section{Filtre 1}
        	Lorem ipsum dolor sit amet\cite{ref1}, consectetur adipiscing elit. Nam lobortis massa eget justo lacinia ultricies. Aenean egestas nunc metus. Pellentesque nibh nibh, placerat eget dui porta, ornare egestas tellus. Donec dictum vel nulla sed feugiat. Donec id bibendum orci, in accumsan nibh. Nunc placerat, sem et sollicitudin elementum, arcu velit scelerisque nibh, et convallis sapien erat a dolor. Pellentesque aliquet lorem erat, eget mollis est dapibus et. Praesent convallis ac nisl sit amet porta.
        \section{Filtre 2}
        	Nam eu sollicitudin massa. Duis sagittis velit mi. Nunc dictum risus ac interdum lacinia\cite{ref2}. Aliquam a fermentum lectus. Praesent dapibus molestie mauris sed vestibulum. Curabitur sodales egestas est a pellentesque. Quisque id vulputate erat, a sagittis turpis. Proin vulputate congue ante, a laoreet orci posuere venenatis. Suspendisse ullamcorper ac dolor nec dapibus.
        \subsection{d�but du filtre 2}
        	Lorem ipsum dolor sit amet, consectetur adipiscing elit. Maecenas rutrum aliquet odio ac volutpat. Suspendisse massa enim, faucibus ut mi dignissim, fringilla ultrices dolor. Quisque bibendum vel nisl eget interdum. Phasellus sed ultricies orci. Aenean egestas risus ut ante porttitor tempus. Suspendisse malesuada, neque ut ullamcorper hendrerit, nibh justo porta purus, quis suscipit neque nisi in nunc. Pellentesque est nisi, porta semper mauris quis, commodo ultricies turpis.
        \subsection{fin du filtre 2}
        	Vivamus posuere malesuada metus, et fringilla felis congue sed. Fusce a nibh in urna molestie viverra vel in est. Donec ipsum nibh, commodo tristique bibendum vitae, consequat quis ipsum. Donec sed diam aliquet velit aliquam viverra. Etiam tincidunt egestas nibh, sit amet facilisis quam vulputate scelerisque. Maecenas nec eros rutrum, tempus ipsum sit amet, tempus dui. Quisque adipiscing varius neque, at egestas erat consequat vitae. Donec sit amet urna iaculis, pharetra arcu in, porttitor quam. Ut nunc ante, tincidunt placerat libero eu, hendrerit lacinia leo. Nam sapien enim, feugiat nec orci eget, porta suscipit diam. Nulla mattis molestie est vitae blandit. Nullam bibendum tempus odio, quis vestibulum quam consectetur at. 
    \chapter{Les MMC}
        \section{Principe}
            \subsection{Graphe}
            	Duis massa elit, vulputate quis lectus vel\cite{ref1}, consectetur ultrices ipsum. Nulla facilisi. Nulla metus lorem, tempus eu arcu non, pulvinar laoreet turpis. Etiam interdum vehicula nunc at convallis. Vestibulum euismod risus at velit interdum suscipit. Vestibulum eu tincidunt massa, sit amet luctus diam. In tempor, nibh sed vulputate mollis, turpis nisl egestas risus, in egestas erat nulla et est. Lorem ipsum dolor sit amet, consectetur adipiscing elit. Ut non tellus fringilla, vehicula nisi eu, commodo lorem. Aliquam erat volutpat. Sed sed tempor ipsum.
            \subsection{Variables}
        \section{Mise en pratique}
        	Suspendisse eget felis quis mauris faucibus dignissim quis vitae dui\cite{ref1}. Maecenas non leo et leo viverra auctor a quis metus. Sed volutpat lectus at sem tincidunt sollicitudin. Maecenas tortor leo, feugiat eget gravida at, scelerisque sit amet nisi. Sed non nunc lorem. Pellentesque ultricies tincidunt ipsum ut sagittis. Maecenas leo tellus, luctus sed gravida vitae\cite{ref2}, dignissim nec sapien. Mauris quis accumsan orci, nec ultricies ante. In sit amet eleifend eros. Ut posuere facilisis risus nec gravida. Nullam consequat sollicitudin mi, nec lacinia quam consectetur et. Integer sagittis nibh metus, at sodales nunc gravida ac. Nunc semper urna nisl, quis congue quam gravida sit amet. Sed a ante imperdiet, luctus purus at, blandit dui.
    \chapter{Le code}
        \section{Python}
        	\SetCodeStyle{Python}
        	\lstinputlisting[language=Python, firstline=2, lastline=37]{exemple.py}
        \section{C/C++}
        	\SetCodeStyle{C++}
        	\lstinputlisting[language=C++, firstline=46, lastline=71]{exemple.cpp}
\bibliographystyle{unsrt} % Pour la biblio
\bibliography{Biblio}

\end{document}