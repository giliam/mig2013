\newpage
\nonewpagechapter*{R�capitulatif}
			Voil� donc l'approche technique que nous avons choisie pour r�aliser ce logiciel de reconnaissance vocale. 
			\\GRAPHIQUE A INCLURE\\
			Les diff�rentes �tapes pr�sent�es plus haut consistent en l'algorithme cod�. 
			Pour cela, nous nous sommes r�partis les t�ches de mani�re � ce que chaque codeur ait en charge une partie de l'algorithme. 
			Il a alors fallu, � la fin, mettre bout � bout les fragments d'algorithme pour les faire fonctionner, ce qui a l� aussi demand� du temps.
			 
Pour la r�partition g�n�rale du code, nous avons confi� la partie la plus sensible, les MMC, � deux codeurs. En effet, il n'�tait pas n�cessaire d'�tre trop nombreux sur cette partie, cela aurait entra�n� une baisse de productivit� et certaines difficult�s de communication. La partie de traitement du signal a r�uni beaucoup plus de codeurs : 2 sur l'enregistrement et le recadrage ; 2 sur l'�chantillonnage et le fen�trage ; 1 sur la transform�e de Fourier ; et 1 sur la transform�e inverse. Enfin, deux autres �l�ves ont cod� l'interface graphique dont nous parlerons � la partie suivante. Les �l�ves restants �taient charg�s respectivement de rassembler les diff�rents �l�ments en un programme unique, de concevoir l'application web et de proc�der � l'�tude �conomique.

